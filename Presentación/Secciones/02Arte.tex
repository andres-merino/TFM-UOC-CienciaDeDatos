%%%%%%%%%%%%%%%%%%%%%%%%%%%%%%%%%%%%%%%%%%%%%%%%%%%%%%%%
\fondo{celeste}
\section{Estado del arte}
\fondo{blanco}
%%%%%%%%%%%%%%%%%%%%%%%%%%%%%%%%%%%%%%%%%%%%%%%%%%%%%%%%

%%%%%%%%%%%%%%%%%%%%%%%%%%%%%%%%%%%%%%%%%%%%%%%%%%%%%%%%
\begin{frame}
    % \frametitle{Reconocimiento de Imágenes y Redes Neuronales}
    \vspace*{-0.5cm}
    \begin{block}{Reconocimiento de Imágenes}
        \begin{itemize}
            \item Clasificación de imágenes.
            \item Localización e identificación de objetos en imágenes.
        \end{itemize}
    \end{block}
    \pause
    \begin{block}{Origen y Evolución de las CNN}
        \begin{itemize}
            \item Inspiración en la corteza visual de animales.
            \item Desarrollo desde Neocognitron (Fukushima, 1982) a LeNet-5 (LeCun, 1989).
            \item Avance impulsado por hardware mejorado y algoritmos optimizados a partir de 2012.
        \end{itemize}
    \end{block}
\end{frame}
%%%%%%%%%%%%%%%%%%%%%%%%%%%%%%%%%%%%%%%%%%%%%%%%%%%%%%%%

%%%%%%%%%%%%%%%%%%%%%%%%%%%%%%%%%%%%%%%%%%%%%%%%%%%%%%%%
\begin{frame}
    \vspace*{-0.5cm}
    \begin{block}{Arquitecturas y Entrenamiento de CNN}
        \begin{itemize}
            \item Ejemplos: AlexNet, VGGNet, GoogLeNet, ResNet.
            \item Entrenamiento: Imagenet, 14 millones de imágenes, 1000 clases.
            \item Entrenamiento con insectos: precisión entre 57\% y 99\%.
        \end{itemize}
    \end{block}
    \pause
    \begin{block}{YOLOv5 y EfficientNet}
        \begin{itemize}
            \item YOLOv5 para detección rápida y precisa.
            \item EfficientNet para clasificación eficiente y escalable.
            \item Ambas arquitecturas adecuadas para desafíos específicos del proyecto.
        \end{itemize}
    \end{block}
\end{frame}
%%%%%%%%%%%%%%%%%%%%%%%%%%%%%%%%%%%%%%%%%%%%%%%%%%%%%%%%

%%%%%%%%%%%%%%%%%%%%%%%%%%%%%%%%%%%%%%%%%%%%%%%%%%%%%%%%
\begin{frame}
    \vspace*{-1cm}
    \begin{block}{Construcción de Redes de Polinización}
        \begin{itemize}
            \item Uso de algoritmos de Monte Carlo y maximización de la esperanza (Young, 2021).
            \item Entrada: matriz de interacciones planta-polinizador.
            \item Salida: matriz de probabilidad de adyacencia (polinizador)
        \end{itemize}
    \end{block}
    \pause
    \begin{block}{Dificultades en la Detección de Insectos}
        \begin{itemize}
            \item Complejidad del fondo.
            \item Tamaño de insectos
            \item Desbalance de clases
        \end{itemize}
    \end{block}
\end{frame}
    