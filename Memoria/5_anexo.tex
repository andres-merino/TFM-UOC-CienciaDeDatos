\chapter{Anexos}
\label{chapter:Anexos}

\section{Repositorio GitHub}
\label{anexo:github}

En el siguiente enlace, se puede acceder al repositorio de GitHub donde se puede encontrar los notebooks con el código necesario para replicar los resultados presentados en este proyecto.
\begin{quote}
    \url{https://github.com/andres-merino/TFM-UOC-CienciaDeDatos}
\end{quote}
De manera específica, los \textit{notebooks} que se pueden encontrar son los siguientes:

\begin{itemize}
    \item \texttt{EfficientNET.ipynb}: \textit{notebook} con el código necesario para realizar el entrenamiento del modelo EfficientNet.
    \item \texttt{YOLOv5-entrenamiento.ipynb}: \textit{notebook} con el código necesario para realizar el entrenamiento del modelo YOLOv5, este fue ejecutado en Google Colab.
    \item \texttt{YOLOv5-prediccion.ipynb}: \textit{notebook} con el código necesario para realizar la predicción del modelo YOLOv5.
    \item \texttt{YOLOv5-prediccion-CrossVal.ipynb}: \textit{notebook} con el código necesario para realizar la predicción del modelo YOLOv5 con validación cruzada.
    \item \texttt{EstimacionRedPolinizadores.ipynb}: \textit{notebook} con el código necesario para realizar la estimación de la red de polinizadores.
    \item \texttt{RedPolinizadores.ipynb.ipynb}: \textit{notebook} con el código necesario para realizar el análisis de la red de polinizadores.
\end{itemize}


\section{Repositorio de pesos de modelos}
\label{anexo:pesos}

En el siguiente enlace, se puede acceder al repositorio particular de OneDrive del autor donde se puede encontrar los pesos de los modelos entrenados en este proyecto.
\begin{quote}
    \url{https://1drv.ms/f/s!AuK-ajP1UCYfoPF4K2pFOxNCGibnJg?e=RXUcK4}
\end{quote}
De manera específica, los pesos que se pueden encontrar son los siguientes:

\begin{itemize}
    \item \texttt{efficientnetb7\_weights.h5}: pesos del modelo EfficientNet.
    \item \texttt{yolov5m\_v01.pt}: pesos del modelo YOLOv5, en su versión \texttt{m}.
    \item \texttt{yolov5l\_v01.pt}: pesos del modelo YOLOv5 en su versión \texttt{l}.
    \item \texttt{yolov5m\_foldn.pt}: pesos del modelo YOLOv5 en el \textit{fold} \texttt{n} de la validación cruzada.
\end{itemize}

