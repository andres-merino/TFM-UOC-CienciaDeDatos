\chapter{Conclusiones y trabajo futuro}
\label{chapter:conclusiones}

Este trabajo ha demostrado con éxito el desarrollo y la aplicación de una red neuronal convolucional (YOLOv5) para el reconocimiento automático de polinizadores en imágenes de flores, un paso significativo en la reconstrucción de redes planta-polinizador. Se ha procesado y analizado un conjunto de datos compuesto por 5445 imágenes, con especial atención en la estandarización del tamaño de las imágenes y la extracción precisa de información relevante mediante archivos XML.

Los resultados obtenidos del modelo YOLOv5 destacan una notable capacidad para detectar y clasificar polinizadores, aunque con ciertas limitaciones en la detección precisa de algunas especies y la diferenciación entre insectos y su entorno. Sin embargo, el uso de un tamaño mínimo de detección ha mejorado significativamente la precisión del modelo.

En lo que respecta a la validación cruzada para el modelo YOLOv5, los resultados obtenidos son alentadores y reflejan la robustez y la capacidad de generalización del modelo. A pesar de ciertas variaciones en la precisión y la exhaustividad entre las categorías, estas fluctuaciones son normales y esperadas en el proceso de validación.

Por otro lado, el modelo EfficientNet no logró un rendimiento satisfactorio, sugiriendo la necesidad de un enfoque más especializado para este tipo de datos.

En cuanto a la reconstrucción de redes de polinizadores, el análisis comparativo entre la red real y la estimada por el modelo ha proporcionado una visión valiosa sobre la eficacia de estas técnicas en la representación precisa de las interacciones en ecosistemas.

Las métricas calculadas sobre las redes muestran que el modelo ha hecho un trabajo notable al reproducir la estructura de la red planta-polinizador y que las diferencias entre las redes real y estimada son, en promedio, insignificantes. Esto sugiere que el modelo puede ser una herramienta confiable para la estimación de redes ecológicas.


Para trabajos futuros, se recomienda:

\begin{itemize}
\item 
    Exploración de Nuevas Técnicas y Modelos: Investigar otras arquitecturas de redes neuronales y métodos de procesamiento de imágenes que puedan mejorar la precisión en la detección y clasificación de polinizadores, especialmente en casos difíciles.
\item 
    Optimización de Modelos Existentes: Mejorar el entrenamiento y ajuste de modelos como EfficientNet para aumentar su eficacia en la detección de polinizadores.
\item 
    Ampliación de la Base de Datos: Incluir más variedades de polinizadores y plantas para enriquecer el conjunto de datos y permitir una generalización más robusta de los modelos.
\item 
    Aplicaciones en Conservación de la Biodiversidad: Explorar cómo estos modelos pueden aplicarse en proyectos de conservación y estudios de biodiversidad para comprender mejor las interacciones planta-polinizador.
\item 
    Generar protocolos para generar la detección de imágenes y la construcción de redes planta-polinizador a partir del trabajo de campo.
\end{itemize}