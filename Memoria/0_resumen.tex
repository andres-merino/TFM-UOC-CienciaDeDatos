%%%%%%%%%%%%%%%%%%%
%%% DEDICATORIA %%%
%%%%%%%%%%%%%%%%%%%
\chapter*{}

\begin{raggedleft}
    \itshape La Ciencia de Datos\\ es un invento de los matemáticos\\ para vender más Matemática.\\
\end{raggedleft}

%%%%%%%%%%%%%%%%%%%
%%% Agradecimientos %%%
%%%%%%%%%%%%%%%%%%%
\chapter*{Agradecimientos}

A mi familia, por sus enseñanzas y los valores que me supieron inculcar. Por las alegrías y problemas que me han llevado a donde ahora estoy. A mis padres, mis hermanas y mi cuñado por el amor y el cariño que han compartido conmigo.

\vspace{\baselineskip}

A mis tutores: Albert Solé-Ribalta, Pau Enric Serra, Javier Borge
Holthoefer, por proponer el problema de este trabajo, por su apoyo y guía durante el desarrollo de este trabajo.

\vspace{\baselineskip}

A Janeth y Mario, quienes me han acompañado en este camino, me motivaron a iniciarlo y me han brindado su apoyo y cariño.

\vspace{\baselineskip}

A mis alumnos, en especial a Melani, Sofía, Martín y Pancho, quienes me ayudan a seguir descubriendo los hermoso del camino de aprender y enseñar. 

\vspace{\baselineskip}

A todos mis amigos (Priss, Eve, Cristian, David, Pau, Mishu,\dots), que cerca o lejos, siempre están ahí para compartir un momento, una charla, un café, una cerveza, una sonrisa, un abrazo, un consejo, una vida.

%%%%%%%%%%%%%%%%
%%% RESUMEN  %%%
%%%%%%%%%%%%%%%%
\chapter*{Abstract}
\addcontentsline{toc}{chapter}{Abstract}

\onehalfspacing

The present work focuses on the construction and training of a convolutional neural network (CNN) with the main objective of detecting pollinators in flower images, which will serve as a basis for the reconstruction of plant-pollinator interaction networks. Specific objectives include the collection and labeling of around 5,000 images of pollinators in two study habitats in Cabrera, Balearic Islands; evaluation of two convolutional neural network architectures (YOLOv5 and EfficientNET) for accurate pollinator detection (in functional groups), training of the selected network, reconstruction of a pollination network from the detections, and performance evaluation using standard network metrics. With the results of this work, we seek to automate the collection and classification of interactions, directly delivering the plant-pollinator interaction network, which offers valuable information both for new research and for decision-making regarding agriculture and conservation.

\vspace{1.5cm}

\paragraph{Keywords:} Image recognition, plant-pollinator network, convolutional neural network.

%%%%%%%%%%%%%%%%
%%% RESUMEN  %%%
%%%%%%%%%%%%%%%%
\chapter*{Resumen}
\addcontentsline{toc}{chapter}{Resumen}

\onehalfspacing

El presente trabajo se centra en la construcción y entrenamiento de una red neuronal convolucional (CNN) con el objetivo principal de detectar polinizadores en imágenes de flores, lo que servirá como base para la reconstrucción de redes de interacción planta-polinizador. Los objetivos específicos incluyen la recopilación y etiquetado de alrededor de 5000 imágenes de polinizadores en dos hábitats de estudio en Cabrera, Islas Baleares; la evaluación de dos arquitecturas de redes neuronales convolucionales (YOLOv5 y EfficientNET) para la detección precisa de polinizadores (en grupos funcionales), el entrenamiento de la red seleccionada, la reconstrucción de una red de polinización a partir de las detecciones y la evaluación del desempeño utilizando métricas estándar sobre la red. Con los resultados de este trabajo, se busca poder automatizar la recolección y clasificación interacciones, entregando directamente la red de interacción planta-polinizador, que ofrezca información valiosa tanto para nuevas investigaciones como para la toma de decisiones en lo referente a agricultura y conservación.



\vspace{1.5cm}

\paragraph{Palabras clave:} Reconocimiento de imágenes, red planta-polinizador, red neuronal convolucional. 